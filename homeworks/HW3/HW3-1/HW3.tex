\documentclass[]{article}
\usepackage{graphicx}
\usepackage[svgnames]{xcolor} 
\usepackage{fancyhdr}

\usepackage{hyperref}
\usepackage{enumitem}
\usepackage[many]{tcolorbox}
\usepackage{listings }
\usepackage[a4paper, total={6in, 8in}]{geometry}
\usepackage{afterpage}
\usepackage{amssymb}
\usepackage{xepersian}
\usepackage[T1]{fontenc}
\usepackage{tikz}
\usepackage[utf8]{inputenc}
\usepackage{PTSerif} 
\usepackage{seqsplit}

\usepackage{listings}
\usepackage{xcolor}
 
\definecolor{codegreen}{rgb}{0,0.6,0}
\definecolor{codegray}{rgb}{0.5,0.5,0.5}
\definecolor{codepurple}{rgb}{0.58,0,0.82}
\definecolor{backcolour}{rgb}{0.95,0.95,0.92}
 
\NewDocumentCommand{\codeword}{v}{
\texttt{\textcolor{blue}{#1}}
}
\lstset{language=C,keywordstyle={\bfseries \color{blue}}}

\lstdefinestyle{mystyle}{
    backgroundcolor=\color{backcolour},   
    commentstyle=\color{codegreen},
    keywordstyle=\color{magenta},
    numberstyle=\tiny\color{codegray},
    stringstyle=\color{codepurple},
    basicstyle=\ttfamily\normalsize,
    breakatwhitespace=false,         
    breaklines=true,                 
    captionpos=b,                    
    keepspaces=true,                 
    numbers=left,                    
    numbersep=5pt,                  
    showspaces=false,                
    showstringspaces=false,
    showtabs=false,                  
    tabsize=2
}

\lstset{style=mystyle}

\settextfont[BoldFont={XB Zar bold.ttf}]{XB Zar.ttf}




\newcommand{\inputsample}[1]{
    ~\\
    \textbf{ورودی نمونه}
    ~\\
    \begin{tcolorbox}[breakable,boxrule=0pt]
        \begin{latin}
            \large{
                #1
            }
        \end{latin}
    \end{tcolorbox}
}

\newcommand{\outputsample}[1]{
    ~\\
    \textbf{خروجی نمونه}

    \begin{tcolorbox}[breakable,boxrule=0pt]
        \begin{latin}
            \large{
                #1
            }
        \end{latin}
    \end{tcolorbox}
}

%%%%%باکس های طراحی شده برای پاسخ نامه ، میتوانید پاسخ را درون باکس قراردهید
\newtcolorbox[auto counter]{solutionbox}{
freelance,
colback=white,
frame code={},
interior titled code={
  \fill[rounded corners=8pt,orange!30]
    (title.south west) --
    (title.south) -- 
    ([yshift=20pt]title.south) --
    ([yshift=20pt,xshift=4cm]title.south) --
    ([xshift=4cm]title.south) --
    (title.south east) {[sharp corners] --
    ([yshift=-6pt]title.south east) -- 
    ([yshift=-6pt]title.south west) } -- cycle;
  \draw[rounded corners=8pt,gray,line width=1pt]
    (title.west|-frame.south west) --
    (title.south west) --
    (title.south) -- 
    ([yshift=20pt]title.south) --
    ([yshift=20pt,xshift=4cm]title.south) --
    ([xshift=4cm]title.south) --
    (title.south east) --
    (title.east|-frame.south east) --
    cycle;
  \node at ([xshift=2cm,yshift=4pt,anchor=south]title.south) 
    {\Large \textbf{پاسخ}};  
  },
title={\mbox{}},
top=12pt,
fontupper=\sffamily\Large,
oversize=0.5cm,
before={\vskip24pt\par\noindent},
after={\par\vskip12pt}
}
\newtcolorbox[auto counter]{solutionboxC}{
freelance,
colback=white,
frame code={},
interior titled code={
  \fill[rounded corners=8pt,orange!30]
    (title.south west) --
    (title.south) -- 
    ([yshift=20pt]title.south) --
    ([yshift=20pt,xshift=4cm]title.south) --
    ([xshift=4cm]title.south) --
    (title.south east) {[sharp corners] --
    ([yshift=-6pt]title.south east) -- 
    ([yshift=-6pt]title.south west) } -- cycle;
  \draw[rounded corners=8pt,gray,line width=1pt]
    (title.west|-frame.south west) --
    (title.south west) --
    (title.south) -- 
    ([yshift=20pt]title.south) --
    ([yshift=20pt,xshift=4cm]title.south) --
    ([xshift=4cm]title.south) --
    (title.south east) --
    (title.east|-frame.south east) --
    cycle;
  \node at ([xshift=2cm,yshift=4pt,anchor=south]title.south) 
    {\Large \textbf{ پاسخ ادامه}};  
  },
title={\mbox{}},
top=12pt,
fontupper=\sffamily\Large,
oversize=0.5cm,
before={\vskip24pt\par\noindent},
after={\par\vskip12pt}
}

\begin{document}


%%% title pages
\begin{titlepage}
\begin{center}
        
\vspace*{0.7cm}

\includegraphics[width=0.4\textwidth]{sharif1.png}\\
\vspace{0.5cm}
\textbf{ \Huge{\emph درس برنامه‌سازی پیشرفته} }\\
\vspace{0.5cm}
\textbf{ \Large{ تمرین سوم بخش اول} }
\vspace{0.2cm}
       
 
      \large \textbf{دانشکده مهندسی کامپیوتر}\\\vspace{0.2cm}
    \large   دانشگاه صنعتی شریف\\\vspace{0.2cm}
       \large   ﻧﯿﻢ سال اول 99-00 \\\vspace{0.2cm}
      \noindent\rule[1ex]{\linewidth}{1pt}
   
   استاد:\\
   \textbf{{وحید سلمانی}}
   
       \vspace{0.20cm}
    مبحث:\\
    \textbf{{گرافیک}}

    \vspace{0.20cm}

   مهلت ارسال:\\
    \textbf{{5 دی}}\\
    \textbf{{ساعت 23:59}}

    \vspace{0.15cm}
ویراستار فنی:\\
    \textbf{{محمد مهدی ابوترابی و پارسا محمدیان}}
\end{center}
\end{titlepage}
%%% title pages


%%% header of pages
\newpage
\pagestyle{fancy}
\fancyhf{}
\fancyfoot{}
\cfoot{\thepage}
\chead{گرافیک}
\rhead{\includegraphics[width=0.1\textwidth]{sharif.png}}
\lhead{تمرین 1.3 برنامه‌سازی پیشرفته}
%%% header of pages




 \Large \textbf{\\\\
به موارد زیر توجه کنید:}

\begin{itemize}[label=$\ast$]
\item به‌ازای هر سوال در سامانه‌ی کوئرا، یک بخش جداگانه برای بارگذاری برنامه‌ی شما وجود دارد. فایل برنامه‌ی خود با پسوند .zip را در بخش مربوط به هر سوال بارگذاری کنید.

\item سوال گرافیک که در بخش اول است در تحویل حضوری بررسی میشود و هنگام آپلود کوئرا آن را داوری نمیکند.
\item هم‌فکری و هم‌کاری در پاسخ به تمرینات اشکالی ندارد و حتی توصیه نیز می‌شود؛ ولی پاسخ ارسالی شما باید حتما توسط خود شما نوشته شده‌باشد. در صورت هم‌فکری در مورد یک سوال، نام افراد دیگر را به‌صورت کامنت در ابتدای کد هر سوال بنویسید.  این نکته رو در نظر بگیرید که هم‌فکری تنها مربوط به بخش ایده سوال هست نه پیاده‌سازی آن و در صورت محرز شدن تقلب برای فرد خاطی بدون مسامحه \emph{ منفی نمره تمرین}
منظور می‌گردد. 
\item شما می‌توانید تمامی سوالات و ابهامات خود را در سایت کوئرا در بخش مشخص‌شده برای این تمرین بپرسید.
\item به‌ازای هر روز تاخیر در ارسال پاسخ هر سوال، 30 درصد از نمره‌ی کسب‌شده‌ی شما در آن سوال کم می‌شود. به عنوان مثال اگر پاسخ یک سوال را با دو روز تاخیر ارسال کنید، فقط 40 درصد از نمره‌ای که برای آن سوال گرفته‌اید برای شما لحاظ خواهد شد.
\item در کل شما می‌توانید سه روز تاخیر بدون کسر نمره داشته باشد.
\item مهلت ارسال تمرین تا ساعت 23:59 روز 5 دی 1399 است.
\end{itemize}




\newpage




\section{ایکس-اوی گرافیگی}


در تمرین سری قبل ایکس-او ای طراحی کردید تا سزار حوصله اش سر نرود! او پس از مدتی بازی کردن از رابط کاربری متنی خسته و حالا از شما می خواهد تا برایش یک بازی ایکس-اوی گرافیکی تهیه کنید. 
همان طور که از توضیحات بند قبل مشخص است، هدف در این سؤال طراحی یک بازی ایکس-او ی گرافیکی است. در این تمرین، یکسری موارد اجباری و یکسری موارد اختیاری برای پیاده‌سازی وجود دارد. موارد اختیاری باعث کسب نمره اضافی علاوه بر نمره کامل سوال می‌شوند.


برای ایده گرفتن و آشنایی بیشتر می توانید به این \href{https://bejofo.net/ttt}{\textcolor{blue}{لینک}} مراجعه کنید. 





\begin{itemize}[label = {$\blacksquare$}]

\item
توجه: برای راحتی کار، در صورتی که کدی که برای تمرین قبل زدید از کیفیت خوبی برخوردار باشد، می‌توانید از بخش‌هایی زیادی از آن کد برای این سؤال هم استفاده کنید ولی لزومی به این کار ندارید و اگر هم بخواهید، می‌توانید کل کد برنامه را دوباره بنویسید.

\setcounter{secnumdepth}{1}

\newpage
\subsection{بخش اجباری (50 نمره)}
\begin{enumerate}


\item
صفحه لاگین (ساختن و ورود به اکانت) –(10 نمره) 

امکاناتی که باید در صفحه مربوط به لاگین و ساخت اکانت موجود باشند:
\begin{itemize}[label = $\circ$]
\item
حذف اکانت (2 نمره)

\item
ساخت اکانت (2 نمره)

\item
لاگین (2 نمره)

\item
تغییر رمز عبور (2 نمره)

\item
خروج از بازی (2 نمره)

\end{itemize}

\item
منوی اصلی (10 نمره)

\begin{itemize}[label = $\circ$]
\item
نیوگیم

\begin{itemize}[label = $\Leftarrow$]
\item
در ابتدای ورود به بازی باید \lr{turn limit} و تعداد دفعات مجاز \lr{Undo} پرسیده شود؛ منظور از \lr{turn limit} چیست، در داک تمرین 2 توضیح داده شده است (4 نمره)
\end{itemize}



\item
اسکوربورد که به همان ترتیب گفته شده در تمرین 2 مرتب بشود و همه موارد تأثیرگذار در مرتب‌سازی (برد و باخت و...) برای هر بازیکن به شکل جدول یا هر شکل مناسب دیگری نمایش داده شوند. (4 نمره)

\item
قابلیت لاگ اوت (2 نمره)

\begin{itemize}[label = $\Leftarrow$]
\item
بعد از لاگ اوت کردن، طبیعتاً باید به منوی قبلی یعنی منوی لاگین و ثبت نام برگردید.
\end{itemize}

\end{itemize}

\item
خود بازی (30 نمره)

\begin{itemize}[label = $\circ$]

\item
همه مورد مربوط به منطق بازی که در داک تمرین ۲ گفته شده‌اند، می‌بایستی پیاده سازی بشوند (12 نمره)

\item
وجود تصویر مناسب و متفاوت برای ایکس و و او ها (5 نمره)

\item
امکان Undo کردن به یک حرکت قبل و مشخص کردن تعداد Undo های مجاز در ابتدای بازی و بازگشت به حالت قبل با انیمیشن (4 نمره)

\item
قرار دادن یک Button یا هر شکل دیگری برای پایان نوبت  (3 نمره)

\item
نمایش پیام‌های مناسب برای هر رویداد (مثلا برد و اتمام بازی - تغییر نوبت و ...)  (3 نمره)
\item
 امکان انصراف از از بازی به وسیله یک Button یا هر شکل دیگری که علاقه دارید. (3 نمره)


\end{itemize}

\end{enumerate}


\subsection{بخش امتیازی (20 نمره)}

هر مورد که پیاده سازی شود، ۴ نمره خواهد داشت. بنابراین نیازی به پیاده سازی تمامی موارد نیست و پس از رسیدن به سقف ۲۰ نمره امتیازی، نمره بیش‌تری برای سایر بخش‌ها داده نمی‌شود.

\begin{enumerate}


\item
امکان زمان‌دار کردن بازی. هر فرد یک زمان‌سنج جدا دارد که با به اتمام رسیدن زمان یک در نوبت وی، آن فرد بازنده شده و فرد دیگر برنده می‌شود. زمان‌سنج هر نفر تنها در نوبت خود وی کار می‌کند.


\item
نگه‌داری لیست کامل حرکت‌ها و توانایی بازگشت به هر کدام از آن‌ها و ادامه دادن بازی از آنجا یعنی به نوعی قابلیت Undo کردن به هر حرکت از حرکات قبلی وجود داشته باشد. طبیعتا تعداد دفعات انجام این کار، همانند شکل ساده Undo محدودیت دارد. در صورت پیاده‌سازی این قابلیت، منطقا نیازی به پیاده سازی Undo یک حرکتی به صورت جداگانه نیست؛ چون پیاده‌سازی این مورد، به خودی خود این موضوع را هم شامل می‌شود.

\item
نشان دادن برد کوچک قابل بازی در هر دست (در صورتی که مطابق قوانین بازی، بازی در همه‌ی جدول های کوچک امکان پذیر بود بردی مشخص نمی‌شود.) 

\item
پخش موسیقی و صداگذاری

این موضوع دو جنبه دارد:
\begin{itemize}[label = $\Leftarrow$]

\item
موسیقی پس زمینه در کل بازی


\item
صداگذاری برای گذاشتن ایکس یا او و کلیک روی دکمه‌ها


\end{itemize}

 موسیقی پس زمینه، 2 نمره و صداگذاری هم 2 نمره دارد.
 

\newpage
\item


فایل (این موضوع دو جنبه دارد هر مورد ۲ نمره)
\begin{itemize}[label = $\Leftarrow$]
	
	\item
ذخیره‌ی اطلاعات اکانت در یک فایل (اطلاعاتی مانند\lr{username}، برد، باخت ...)
	
	
	\item
خواندن اطلاعات اکانت از روی فایل
	
	
\end{itemize}
در واقع در این صورت با terminate شدن برنامه شما باید اطلاعات کاربر ها باقی بماند.

\item
زیبایی بازی




\end{enumerate}

\setcounter{secnumdepth}{6}

\end{itemize}


\end{document}









